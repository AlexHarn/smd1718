\documentclass[a4paper, 11pt]{article}
\usepackage{comment} 
\usepackage{fullpage}
\usepackage{amsmath} 
\usepackage{amssymb} 
\usepackage{mathtools}
\usepackage{fontspec}
\defaultfontfeatures{Ligatures=TeX}
\usepackage{xfrac}
\usepackage{icomma}
\usepackage[section,below]{placeins}
\usepackage[labelfont=bf,font=small,width=0.9\textwidth]{caption}
\usepackage{subcaption}
\usepackage{graphicx}
\usepackage{grffile}
\usepackage{float}
\floatplacement{figure}{htbp}
\floatplacement{table}{htbp}
\usepackage{booktabs}
\usepackage{hyperref}
\usepackage[ngerman]{babel}
\begin{document}
\noindent
%\centerline{\small{\textsc{Technische Universität Dortmund}}} \\
\large{\textbf{<+Nr+>. Übungsblatt zur Vorlesung \hfill WS 2017/2018 \\
Statistische Methoden der Datenanalyse \hfill Prof. W. Rhode}} \\
Annika Burkowitz, Sebastian Bange, Alexander Harnisch \\
\noindent\makebox[\linewidth]{\rule{\textwidth}{0.4pt}}

\section*{Aufgabe <+Nr+>}

\end{document}
